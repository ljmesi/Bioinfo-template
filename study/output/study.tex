% Options for packages loaded elsewhere
\PassOptionsToPackage{unicode}{hyperref}
\PassOptionsToPackage{hyphens}{url}
\PassOptionsToPackage{dvipsnames,svgnames*,x11names*}{xcolor}
%
\documentclass[
]{article}
\usepackage{lmodern}
\usepackage{amssymb,amsmath}
\usepackage{ifxetex,ifluatex}
\ifnum 0\ifxetex 1\fi\ifluatex 1\fi=0 % if pdftex
  \usepackage[T1]{fontenc}
  \usepackage[utf8]{inputenc}
  \usepackage{textcomp} % provide euro and other symbols
\else % if luatex or xetex
  \usepackage{unicode-math}
  \defaultfontfeatures{Scale=MatchLowercase}
  \defaultfontfeatures[\rmfamily]{Ligatures=TeX,Scale=1}
\fi
% Use upquote if available, for straight quotes in verbatim environments
\IfFileExists{upquote.sty}{\usepackage{upquote}}{}
\IfFileExists{microtype.sty}{% use microtype if available
  \usepackage[]{microtype}
  \UseMicrotypeSet[protrusion]{basicmath} % disable protrusion for tt fonts
}{}
\makeatletter
\@ifundefined{KOMAClassName}{% if non-KOMA class
  \IfFileExists{parskip.sty}{%
    \usepackage{parskip}
  }{% else
    \setlength{\parindent}{0pt}
    \setlength{\parskip}{6pt plus 2pt minus 1pt}}
}{% if KOMA class
  \KOMAoptions{parskip=half}}
\makeatother
\usepackage{xcolor}
\IfFileExists{xurl.sty}{\usepackage{xurl}}{} % add URL line breaks if available
\IfFileExists{bookmark.sty}{\usepackage{bookmark}}{\usepackage{hyperref}}
\hypersetup{
  pdftitle={Example title name},
  pdfauthor={Author name},
  pdfsubject={This subject text can be viewed in pdf properties.},
  pdfkeywords={these, keywords, are visible in, pdf properties},
  colorlinks=true,
  linkcolor=MidnightBlue,
  filecolor=Maroon,
  citecolor=Aquamarine,
  urlcolor=NavyBlue,
  pdfcreator={LaTeX via pandoc}}
\urlstyle{same} % disable monospaced font for URLs
\usepackage[margin=1.0in]{geometry}
\usepackage{color}
\usepackage{fancyvrb}
\newcommand{\VerbBar}{|}
\newcommand{\VERB}{\Verb[commandchars=\\\{\}]}
\DefineVerbatimEnvironment{Highlighting}{Verbatim}{commandchars=\\\{\}}
% Add ',fontsize=\small' for more characters per line
\usepackage{framed}
\definecolor{shadecolor}{RGB}{248,248,248}
\newenvironment{Shaded}{\begin{snugshade}}{\end{snugshade}}
\newcommand{\AlertTok}[1]{\textcolor[rgb]{0.94,0.16,0.16}{#1}}
\newcommand{\AnnotationTok}[1]{\textcolor[rgb]{0.56,0.35,0.01}{\textbf{\textit{#1}}}}
\newcommand{\AttributeTok}[1]{\textcolor[rgb]{0.77,0.63,0.00}{#1}}
\newcommand{\BaseNTok}[1]{\textcolor[rgb]{0.00,0.00,0.81}{#1}}
\newcommand{\BuiltInTok}[1]{#1}
\newcommand{\CharTok}[1]{\textcolor[rgb]{0.31,0.60,0.02}{#1}}
\newcommand{\CommentTok}[1]{\textcolor[rgb]{0.56,0.35,0.01}{\textit{#1}}}
\newcommand{\CommentVarTok}[1]{\textcolor[rgb]{0.56,0.35,0.01}{\textbf{\textit{#1}}}}
\newcommand{\ConstantTok}[1]{\textcolor[rgb]{0.00,0.00,0.00}{#1}}
\newcommand{\ControlFlowTok}[1]{\textcolor[rgb]{0.13,0.29,0.53}{\textbf{#1}}}
\newcommand{\DataTypeTok}[1]{\textcolor[rgb]{0.13,0.29,0.53}{#1}}
\newcommand{\DecValTok}[1]{\textcolor[rgb]{0.00,0.00,0.81}{#1}}
\newcommand{\DocumentationTok}[1]{\textcolor[rgb]{0.56,0.35,0.01}{\textbf{\textit{#1}}}}
\newcommand{\ErrorTok}[1]{\textcolor[rgb]{0.64,0.00,0.00}{\textbf{#1}}}
\newcommand{\ExtensionTok}[1]{#1}
\newcommand{\FloatTok}[1]{\textcolor[rgb]{0.00,0.00,0.81}{#1}}
\newcommand{\FunctionTok}[1]{\textcolor[rgb]{0.00,0.00,0.00}{#1}}
\newcommand{\ImportTok}[1]{#1}
\newcommand{\InformationTok}[1]{\textcolor[rgb]{0.56,0.35,0.01}{\textbf{\textit{#1}}}}
\newcommand{\KeywordTok}[1]{\textcolor[rgb]{0.13,0.29,0.53}{\textbf{#1}}}
\newcommand{\NormalTok}[1]{#1}
\newcommand{\OperatorTok}[1]{\textcolor[rgb]{0.81,0.36,0.00}{\textbf{#1}}}
\newcommand{\OtherTok}[1]{\textcolor[rgb]{0.56,0.35,0.01}{#1}}
\newcommand{\PreprocessorTok}[1]{\textcolor[rgb]{0.56,0.35,0.01}{\textit{#1}}}
\newcommand{\RegionMarkerTok}[1]{#1}
\newcommand{\SpecialCharTok}[1]{\textcolor[rgb]{0.00,0.00,0.00}{#1}}
\newcommand{\SpecialStringTok}[1]{\textcolor[rgb]{0.31,0.60,0.02}{#1}}
\newcommand{\StringTok}[1]{\textcolor[rgb]{0.31,0.60,0.02}{#1}}
\newcommand{\VariableTok}[1]{\textcolor[rgb]{0.00,0.00,0.00}{#1}}
\newcommand{\VerbatimStringTok}[1]{\textcolor[rgb]{0.31,0.60,0.02}{#1}}
\newcommand{\WarningTok}[1]{\textcolor[rgb]{0.56,0.35,0.01}{\textbf{\textit{#1}}}}
\usepackage{longtable,booktabs}
% Correct order of tables after \paragraph or \subparagraph
\usepackage{etoolbox}
\makeatletter
\patchcmd\longtable{\par}{\if@noskipsec\mbox{}\fi\par}{}{}
\makeatother
% Allow footnotes in longtable head/foot
\IfFileExists{footnotehyper.sty}{\usepackage{footnotehyper}}{\usepackage{footnote}}
\makesavenoteenv{longtable}
\usepackage{graphicx}
\makeatletter
\def\maxwidth{\ifdim\Gin@nat@width>\linewidth\linewidth\else\Gin@nat@width\fi}
\def\maxheight{\ifdim\Gin@nat@height>\textheight\textheight\else\Gin@nat@height\fi}
\makeatother
% Scale images if necessary, so that they will not overflow the page
% margins by default, and it is still possible to overwrite the defaults
% using explicit options in \includegraphics[width, height, ...]{}
\setkeys{Gin}{width=\maxwidth,height=\maxheight,keepaspectratio}
% Set default figure placement to htbp
\makeatletter
\def\fps@figure{htbp}
\makeatother
\setlength{\emergencystretch}{3em} % prevent overfull lines
\providecommand{\tightlist}{%
  \setlength{\itemsep}{0pt}\setlength{\parskip}{0pt}}
\setcounter{secnumdepth}{5}
\pagestyle{headings}
% https://github.com/rstudio/rmarkdown/issues/337
\let\rmarkdownfootnote\footnote%
\def\footnote{\protect\rmarkdownfootnote}

% https://github.com/rstudio/rmarkdown/pull/252
\usepackage{titling}
\setlength{\droptitle}{-2em}

\pretitle{\vspace{\droptitle}\centering\huge}
\posttitle{\par}

\preauthor{\centering\large\emph}
\postauthor{\par}

\predate{\centering\large\emph}
\postdate{\par}
\newlength{\cslhangindent}
\setlength{\cslhangindent}{1.5em}
\newenvironment{cslreferences}%
  {\setlength{\parindent}{0pt}%
  \everypar{\setlength{\hangindent}{\cslhangindent}}\ignorespaces}%
  {\par}

\title{Example title name}
\usepackage{etoolbox}
\makeatletter
\providecommand{\subtitle}[1]{% add subtitle to \maketitle
  \apptocmd{\@title}{\par {\large #1 \par}}{}{}
}
\makeatother
\subtitle{Some example subtitle}
\author{Author name}
\date{2020-03-17}

\begin{document}
\maketitle

{
\hypersetup{linkcolor=NavyBlue}
\setcounter{tocdepth}{2}
\tableofcontents
}
\clearpage

\hypertarget{introduction}{%
\section{Introduction}\label{introduction}}

Here is introduction to this example document. This document is written in R Studio server (RStudio Team \protect\hyperlink{ref-RStudioTeam2020}{2020}) and running in R Docker container (Boettiger and Eddelbuettel \protect\hyperlink{ref-Boettiger2017}{2017}).

\hypertarget{the-file-structure}{%
\subsection{The file structure}\label{the-file-structure}}

The file structure of this default project is inspired by \href{https://github.com/SchlossLab/new_project}{SchlossLabs new\_project github repository}, article by Wilson et al. (\protect\hyperlink{ref-Wilson2017}{2017}) and

Here below is a simple schema of it:

\begin{verbatim}
project
|- README.md                                    # the top level description of content (this doc)
|- CONTRIBUTING                                 # instructions for how to contribute to your project
|- LICENSE                                      # the license for this project
|- CITATION                                     # instructions on how to cite the work
|
|- study/
| |- header.sty                                 # LaTeX header file to format pdf version of manuscript
| |- bibliography.bib                           # BibTeX formatted references
| |- csl/                                       # csl files to format references
| |- study.Rmd                                  # executable Rmarkdown for this study, if applicable
| +- output/                                    # images and other graphics for the presentation
| | |- study.md                                 # Markdown (GitHub) version of the *.Rmd file
| | |- study.tex                                # TeX version of *.Rmd file
| | |- study.pdf                                # PDF version of *.Rmd file
| | |- study.html                               # html version of *.Rmd file
|
|- data                                         # raw and primary data, are not changed once created
| |- references/                                # reference files to be used in analysis
| |- raw/                                       # raw data, will not be altered
| |- processed/                                 # cleaned data, will not be altered once created;
|                                               # will be committed to repo
|
|- docker                                       # Files related to docker virtualisation
| |- Dockerfile                             # Dockerfile defining the development environment
| |- add_shiny.sh                           # These four files below are all configuration files
| |- disable_auth_rserver.conf    # used in building an image from the Dockerfile
| |- pam-helper.sh                          # obtained from Rocker projects github page:
| |- userconf.sh                            # https://github.com/rocker-org/rocker-versioned/tree/master/rstudio
|                                 # The Dockerfile has also heavily loaned code from the Rocker project:
|                                 # https://www.rocker-project.org/
|
|- presentations                                # presentations about the project 
| |- _output.yaml                               # shared configurations for all presentations
| |- style.css                                  # css for modifying features in presentation
| |- presentation.Rmd                       # Rrevealjs presentation
| |- presentation.html                      # rendered html of presentation.Rmd
| |- images/                                    # images and other graphics for the presentation
|
|- scripts/                                     # any programmatic code
|
|- results                                      # all output from workflows and analyses
| |- tables/                                    # tables and other tabular data
| |- figures/                                   # graphs, likely designated for manuscript figures
| |- pictures/                                  # diagrams, images, and other non-graph graphics
|
|- exploratory/                                 # exploratory data analysis for study
| |- notebook/                                  # preliminary analyses
| |- scratch/                                   # temporary files that can be safely deleted or lost
|
|- Snakefile                                    # executable Snakefile for this study, if applicable
\end{verbatim}

\clearpage

\hypertarget{some-chapter}{%
\section{Some chapter}\label{some-chapter}}

I prefer that 1. level headings start from a new page and try not to have heading levels higher than 2.

Here is an example Figure 1.

\begin{figure}
\centering
\includegraphics[width=0.67\textwidth,height=\textheight]{../presentations/images/example.jpg}
\caption{Here is an example figure.}
\end{figure}

\hypertarget{here-is-a-sub-chapter}{%
\subsection{Here is a sub chapter}\label{here-is-a-sub-chapter}}

Footnotes can be inserted with a simple syntax\footnote{Here is a foot note text.}. The following Table 1 is an example.

\begin{longtable}[]{@{}lll@{}}
\caption{Here is an example table.}\tabularnewline
\toprule
No & Column 1 & Column 2\tabularnewline
\midrule
\endfirsthead
\toprule
No & Column 1 & Column 2\tabularnewline
\midrule
\endhead
1 & Some & Data\tabularnewline
2 & Here too & and here\tabularnewline
3 & and finally & here\tabularnewline
\bottomrule
\end{longtable}

In order to convert this Rmarkdown document to both html and pdf format, you should move this section in the YAML header:

\begin{verbatim}
  bookdown::pdf_document2:
    keep_tex: no
    latex_engine: xelatex
    # For more pandoc args see:
    # https://pandoc.org/MANUAL.html
    pandoc_args: ["--top-level-division=section",
                  "-V", "documentclass=article",
                  "-V", "linkcolor=MidnightBlue",
                  "-V", "citecolor=Aquamarine",
                  "-V", "urlcolor=NavyBlue",
                  "-V", "toccolor=NavyBlue",
                  "-V", "pagestyle=headings"]
\end{verbatim}

above this section:

\begin{verbatim}
  bookdown::html_document2:
    highlight: tango
    theme: yeti
    split_by: none # only generate a single output page
    self_contained: TRUE
    toc: yes
    toc_float: 
      collapsed: FALSE
      smooth_scroll: TRUE
      print: FALSE
    code_folding: hide
    number_sections: TRUE
    code_download: TRUE
    pandoc_args: ["--top-level-division=section",
              "-V", "documentclass=report"]
\end{verbatim}

but below the line with \texttt{output:}. This is quite hacky but it works for now\ldots{}

\clearpage

\hypertarget{session-info}{%
\section{Session info}\label{session-info}}

\begin{Shaded}
\begin{Highlighting}[]
\KeywordTok{sessionInfo}\NormalTok{()}
\end{Highlighting}
\end{Shaded}

\begin{verbatim}
R version 3.6.1 (2019-07-05)
Platform: x86_64-pc-linux-gnu (64-bit)
Running under: Debian GNU/Linux 9 (stretch)
...
\end{verbatim}

\clearpage

\hypertarget{references}{%
\section*{References}\label{references}}
\addcontentsline{toc}{section}{References}

\hypertarget{refs}{}
\begin{cslreferences}
\leavevmode\hypertarget{ref-Boettiger2017}{}%
Boettiger, Carl, and Dirk Eddelbuettel. 2017. ``An Introduction to Rocker: Docker Containers for R.'' \emph{R Journal} 9 (2): 527--36. \url{https://rocker-project.org.\%20http://arxiv.org/abs/1710.03675}.

\leavevmode\hypertarget{ref-RStudioTeam2020}{}%
RStudio Team. 2020. ``RStudio: Integrated Development Environment for R.'' Boston, MA: RStudio, Inc. \url{http://www.rstudio.com/}.

\leavevmode\hypertarget{ref-Wilson2017}{}%
Wilson, Greg, Jennifer Bryan, Karen Cranston, Justin Kitzes, Lex Nederbragt, and Tracy K Teal. 2017. ``Good enough practices in scientific computing.'' Edited by Francis Ouellette. \emph{PLOS Computational Biology} 13 (6): e1005510. \url{https://doi.org/10.1371/journal.pcbi.1005510}.
\end{cslreferences}

\end{document}
